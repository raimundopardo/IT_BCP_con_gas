
\usepackage{helvet} % Fuente Helvética para los informes técnicos.

\renewcommand{\familydefault}{\sfdefault}

\usepackage[T1]{fontenc}

\usepackage[utf8]{inputenc}

\usepackage{fancyhdr}
%\pagestyle{fancy}  %este comando genera errores.

\usepackage{totcount} %necesario para totalizacion de destinatarios en lista de distribucion

\usepackage{graphicx}

\usepackage{pstricks}

% afterpage se utiliza para evitar acumulacion de figuras sin procesar
\usepackage{afterpage}

\usepackage{amsmath}

\usepackage{setspace}

\usepackage{units}



\usepackage{url}

\usepackage{euscript}

\usepackage{textcomp}

\usepackage[noconfig]{unitsdef}

\usepackage{psfrag}

\usepackage{float}

\usepackage{parskip} % mejora el aspecto de la separacion entre párrafos



\usepackage{titlesec} % necesario para cambiar el estilo de los títulos de seccion y referencias

\usepackage[spanish]{babel}

\usepackage{fancyhdr}

\usepackage{ccaption}

\captiondelim{. }

\captionnamefont{\normalfont\normalsize}

\captiontitlefont{\normalfont\normalsize}

\usepackage{sectsty}

\usepackage{enumitem}

\usepackage{tocloft}

\usepackage{remreset} %necesario para eliminar en número de sección en la numeracipon de figuras y tablas

\usepackage[spanish]{babel}

\usepackage{csquotes}

\usepackage[citestyle=authoryear,language = spanish, bibstyle=authoryearint1,backend=biber,abbreviate=true,maxcitenames=2]{biblatex}




\renewcommand{\normalfont}{\sffamily}

%Espaciado de linea de una linea dentro de los párrafos

\renewcommand*{\baselinestretch}{1.}

%espacio entre párrafos.

\setlength{\parskip}{1 em}

%Se eliminan las sangrias

\setlength{\parindent}{0in}


\titleformat{\chapter}{\normalfont\bfseries\uppercase}{\thechapter.}{1 ex}{}

\titlespacing*{\chapter}{0 ex}{0 ex}{0 ex}

\titleclass{\section}{straight} %este comando es necesatio para que \titleformat \section funcione cuando se utilice a mitad de pagina.

\titleformat{\section}{\normalfont\bfseries}{\thesection.}{1 ex}{}
%
\titlespacing*{\section}{0 ex}{0 ex}{0 ex}
%%


\makeatletter


                                   
% los siguientes comandos eliminan los números de la seccion en los títulos de las figuras y tablas                                   
                                   

 \@removefromreset{figure}{chapter}
 
 
 \renewcommand{\thefigure}{\arabic{figure}}
 
 
  \@removefromreset{table}{chapter}
  
  
  \renewcommand{\thetable}{\arabic{table}}
                                      

\makeatother


\renewcommand{\tablename}{\normalfont{Tabla}}

\renewcommand{\figurename}{\normalfont{Figura}}



% EL valor por omisión de patentus en español es Pat. americana. Esta denominación se cambia en la siguiente definición (biblatex):
 
\DefineBibliographyStrings{spanish}{ %
patentus = {Patente US}}

%Formato Intevep

\headheight = 14pt


\voffset = -20pt

\oddsidemargin = -2pt

\evensidemargin = -2pt

\textheight = 617pt

\textwidth = 470pt

\parindent = 0pt




\renewcommand{\floatpagefraction}{0.7}







\renewcommand{\headrulewidth}{0pt}  %elimina la línea horizontal en los encabezados de página

\newcommand{\NIT}{\normalfont{INT-14224,2013}}

\newcommand{\Intevep}{\small \textbf{Intevep, S.A.} Centro de Investigación y Desarrollo. Filial de Petróleos de Venezuela, S.A.}

\newcounter{numerocopia}

\setcounter{numerocopia}{0}

\regtotcounter{numerocopia}


\newcommand{\destinatario}[3]{

\stepcounter{numerocopia}

\begin{tabbing}

xxxxxxxxxxxxxxxxxxxxxxxxxxxxxxxxxxxxxxxxxxxxxxxxxxxxxxxxxxxxxxxxxxxxx \=      \kill

\normalfont{#1}                          \> \normalfont{\thenumerocopia(\total{numerocopia})} \\
\normalfont{#2  }                    \\
\normalfont{Responsable: #3 }

\end{tabbing}
}

\newenvironment{distribucion}{%

\begin{center} {\normalfont\textbf{LISTA DE DISTRIBUCI\'ON}} \end{center}

\begin{tabbing}
    
    
    xxxxxxxxxxxxxxxxxxxxxxxxxxxxxxxxxxxxxxxxxxxxxxxxxxxxxxxxxxxxxxxxx \=      \kill
    
                                                                                 \>\normalfont{No.} \=\normalfont{de copias}  \\

\end{tabbing}}
{\newpage}


\newenvironment{sumario}{\begin{center} 
\normalfont\textbf{SUMARIO}
\end{center}

\addcontentsline{toc}{section}{\bfseries SUMARIO}

\normalfont}{\newpage}

\newcommand{\contenidopreliminar}{

%\singlespacing

\frontmatter

\setcounter{page}{3} % es necesario fijar la primera página en 3.

\pagestyle{fancy}

\lhead[\NIT]{\NIT}

\rhead[]{}
\cfoot{\normalfont{\thepage \\ \Intevep}}
}

%Se eliminan los puntos de las tablas de contenido y figuras

\renewcommand{\cftdotsep}{\cftnodots}  % comando de tocloft para eliminar los puntos separadores entre seccion y página







% Tabla de contenido:

\newcommand{\tabladecontenido}{

\normalfont



\fancypagestyle{plain}  % Es necesario colocar este comando para que aparezca el heading en la página de tabla de contenido.

\lhead[\NIT]{\NIT}

\cfoot{\normalfont{\thepage \\ \Intevep}}
    
\renewcommand*{\contentsname}{\normalsize \centerline{TABLA DE CONTENIDO}}  %Por alguna razón, se requiere el * para cambiar \contentsname. Ver libro de Salinas, p.95.    
 

\setlength{\cftbeforetoctitleskip}{-1cm} % comando del package tocloft para reducir el espacio antes del título de tabla de contenido

\setlength{\cftaftertoctitleskip}{24pt} % comando del package tocloft para reducir el espacio antes del título de tabla de contenido

%Los siguientes dos comandos de tocloloft reducen a cero la indentacion en la tabla de contenido

\setlength{\cftchapindent}{0mm}

\setlength{\cftsecindent}{0mm}


\renewcommand{\cftchapaftersnum}{.}

\renewcommand{\cftsecaftersnum}{.}

\renewcommand{\cftchappagefont}{\normalfont} % coloca el número de la página en tipo normal.

\renewcommand{\cftsecfont}{\bfseries} % se utiliza bolf para las secciones

\setlength{\cftbeforechapskip}{1em} % se fija la separación entre entradas

\setlength{\cftbeforesecskip}{1em}

\clearpage  
 
   
\tableofcontents    
}

% Se agregan las líneas con el rótulo de páginas en las tablas de contenido, figuras y tablas.

\addtocontents{toc}{\protect \rightline{Página}}

\addtocontents{lof}{\protect  \rightline{Página}}

\addtocontents{lot}{\protect  \rightline{Página}}


% Se utliliza el paquete babel español para redefinir listfiguresname

\renewcommand*{\spanishlistfigurename}{\normalsize \centerline{LISTA DE ILUSTRACIONES}}



\newcommand{\listadeilustraciones}{

\normalfont

\setlength{\cftbeforeloftitleskip}{-1cm} % comando del package tocloft para reducir el espacio antes del título de tabla de contenido

\setlength{\cftafterloftitleskip}{24pt} % comando del package tocloft para reducir el espacio antes del título de tabla de contenido

\clearpage

\setlength{\cftfigindent}{0mm}



\setlength{\cftfignumwidth}{4em}

\renewcommand{\cftfigpresnum}{Fig.\thinspace}

\renewcommand{\cftfigaftersnum}{.}

\setlength{\cftbeforefigskip}{1em} % se fija la separación entre entradas

\listoffigures

\addcontentsline{toc}{section}{\textbf{LISTA DE ILUSTRACIONES}}



}

\renewcommand*{\spanishlisttablename}{\normalsize \centerline{LISTA DE TABLAS}}  %Por alguna razón, se requiere el * para cambiar \contentsname. Ver libro de Salinas, p.95.    

\newcommand{\listadetablas}{

\normalfont

\setlength{\cftbeforelottitleskip}{-1cm} % comando del package tocloft para reducir el espacio antes del título de tabla de contenido

\setlength{\cftafterlottitleskip}{24pt} % comando del package tocloft para reducir el espacio antes del título de tabla de contenido

\clearpage

\setlength{\cfttabnumwidth}{4em}

\setlength{\cfttabindent}{0mm}

\renewcommand{\cfttabpresnum}{Tabla\thinspace}

\renewcommand{\cfttabaftersnum}{.}


\setlength{\cftbeforetabskip}{1em} % se fija la separación entre entradas

\listoftables

\addcontentsline{toc}{section}{\textbf{LISTA DE TABLAS}}

\newpage

}





\newcommand{\contenidoprincipal}{

\mainmatter

% Se redefine el estilo plain para utilizarlo en la primera página de texto.
\fancypagestyle{plain}{%
\fancyhf{} % clear all header and footer fields
\fancyhead[LO]{\NIT}
\fancyfoot[C]{\Intevep}} % except the center

\pagestyle{fancy}

\rhead[\NIT]{\thepage}

\lhead[\thepage]{\NIT}

\cfoot{\Intevep}

\thispagestyle{plain} % Este comando es necesario para suprimir el número en la página 1.
}

% % % % % % % %

\newcommand{\referencias}{

\fancypagestyle{plain}  % Es necesario colocar este comando para que aparezca el heading en la página de referencias.

\vspace*{-5cm} % este comando debe ir antes de \bibliographystyle. En caso contrario, no funciona.

\titleformat{\chapter}{\normalfont}{}{}{}{}

\titlespacing*{\chapter}{0pt}{0pt}{12pt}



\printbibliography[title=\textbf{REFERENCIAS}]

\addcontentsline{toc}{section}{\textbf{REFERENCIAS}}

}


\newcommand{\fuente}[1]{\small Fuente: #1}

