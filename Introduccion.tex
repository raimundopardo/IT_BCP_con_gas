\chapter{INTRODUCCIÓN}


Las bombas de cavidades progresivas (BCP) son utilizadas ampliamente en el levantamiento de petróleo, con cientos de instalaciones en Venezuela y miles a nivel mundial. Salvo casos excepcionales, las BCP manejan una mezcla bifásica de gas y crudo, condición que es característica de la industria petrolera. En este sentido, es de importancia conocer el comportamiento de las BCP manejando mezclas gas-líquido.

Hasta el día de hoy, se le ha prestado poca atención al comportamiento fluidodinámico y termodinámico del proceso de bombeo en las BCP, tanto de líquidos como de mezclas compresibles. La documentación técnica de las BCP para subsuelo solo contienen datos básicos sobre eficiencia volumétrica y potencia requerida para el bombeo de agua. Aspectos del proceso de bombeo tales como efectos de la viscosidad del líquido,  presencia de una fase gaseosa y ajuste entre rotor y estator no están documentados. El impacto de estas variables sobre la vida útil de la bomba y su desempeño global es desconocido.

Por otra parte, en los últimos años se ha venido observando un incremento en el volumen de gas asociado a la producción en los yacimiento de la Faja Petrolífera del Orinoco (FPO). Conocer el impacto de este cambio en las características de los fluidos bombeados por los sistemas BCP es de importancia para el mantenimiento de las operaciones de producción, diagnóstico acertado de fallas prematuras y procura de nuevas bombas.

El presente informe contiene un resumen de la literatura en materia de BCP bombeando mezclas bifásicas y otras condiciones de importancia para la producción de petróleo tales como viscosidad y ajuste rotor-estator. La búsqueda de referencias está limitada a publicaciones en revistas periódicas, congresos, trabajos de grado y patentes. En general, la información técnica en otras fuentes tales como manuales, monografías, catálogos y páginas web de los fabricantes es prácticamente nula.




