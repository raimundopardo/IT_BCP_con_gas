En pruebas de campo y de laboratorio se ha demostrado que la fracción de gas libre reduce la eficiencia volumétrica de las bombas de cavidades progresivas (BCP) y se crean esfuerzos localizados, además de un incremento en la temperatura del elastómero que conlleva a perdidas de las propiedades mecánicas del mismo, lo cual finalmente reduce la vida útil de las BCP. Además se ha encontrado que la forma de la distribución de presión a lo largo de la bomba también influye en la vida útil de la misma y es función del ajuste rotor estator, viscosidad del fluido, RPM y GVF.

Por ello algunos fabricantes están buscando la manera de minimizar el impacto de este problema. A continuación se muestran algunas publicaciones técnicas donde se identifican algunos hallazgos importantes para entender e intentar minimizar dicho impacto.

Actualmente en la Faja Petrolífera del Orinoco (FPO), los pozos producen con altas fracciones de gas libre en la entrada de la bomba, cabe destacar que el método de Levantamiento Artificial (LA) más empleado es el de BCP. Por lo tanto esta línea de investigación resulta atractiva y necesaria para producir de manera eficiente la FPO.